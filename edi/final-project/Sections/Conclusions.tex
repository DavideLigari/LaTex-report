\section{Conclusions}
This study examined the DNS amplification attack and its impact on network resources, specifically focusing on DNS queries and Ping measurements.
The attack was successfully executed, resulting in a significant increase in DNS query times and latency. Despite the attack's effectiveness,
it did not lead to a complete denial of service, indicating that the targeted system had efficient resource management mechanisms in place.\\
However, the attack did have a notable side effect on the server resources, particularly on CPU utilization.
The increased query traffic caused a significant rise in CPU usage, highlighting the strain on the DNS server.
This finding emphasizes the importance of considering the impact on server resources when mitigating DNS reflection attacks.\\
Overall, this study underscores the significance of implementing effective mitigation strategies to protect against DNS amplification attacks.
By employing a combination of proactive, detection, and resilience measures, network administrators can enhance network
resilience and mitigate the detrimental consequences of such attacks.\\
Future research should continue to explore and develop innovative approaches to combat evolving DNS amplification attack techniques
and safeguard network infrastructures.