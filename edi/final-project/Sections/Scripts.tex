\section{Scripts}
\subsubsection*{Spoofing}
A custom script based on the \textit{Scapy} library was employed to conduct a Distributed Denial of Service (DDoS) attack.
The identification of active hosts within the target network was facilitated by performing a ping sweep on a specific network range.
The delivery of ARP request packets to all hosts was ensured by sending them with a broadcast destination MAC address,
utilizing the packet crafting capabilities of the Scapy library.
The IP addresses of the active hosts were extracted by analyzing the received responses within a specified timeout period.
This information allowed for the identification of potential targets for subsequent stages of the DDoS attack.
The technique employed in this stage served as an initial reconnaissance step in the attack, providing valuable insights
into the composition of the network and the active hosts that could be further exploited to disrupt the target's services.

\subsubsection*{DoS Script}
A customized script utilizing the \textit{Scapy} and \textit{dnspython} libraries was employed to execute a Distributed Denial of Service (DDoS) attack.
The primary objective of the attack was to flood a target host with a massive number of DNS query packets. The script allowed for the specification
of the spoofed IP address, which would appear as the source of the attack, through its command-line arguments.
This technique aimed to deceive the target and potentially implicate the spoofed IP in the attack.
Various parameters, such as the IP address and port number of the target DNS server, the domain name to query,
and the DNS flags to manipulate, could be configured using the script.
By manipulating these flags, the DNS message could be customized according to the attack goals.
Additionally, the script provided the option to specify the number of threads to utilize,
with each thread being responsible for generating a specific number of DNS requests.
This multi-threaded approach amplified the impact of the attack by concurrently inundating the target with a high volume of DNS queries.
The cumulative effect was an overwhelming amount of traffic directed towards the specified spoofed IP,
resulting in the disruption of its normal operations and potential denial of service.