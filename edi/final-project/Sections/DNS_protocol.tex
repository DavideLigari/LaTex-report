\section{The DNS protocol}
The Domain Name System (DNS) protocol is a hierarchical and distributed naming system for computers, services, or any resources connected to the Internet or a private network. It translates domain names, such as example.com, into IP addresses, allowing devices to locate and communicate with each other. DNS operates on a client-server model, with DNS clients sending queries to DNS servers to resolve domain names.

\subsection*{Hierarchy}
The DNS protocol is organized hierarchically, with a global system of interconnected DNS servers. At the top of the hierarchy are the root DNS servers, followed by top-level domain (TLD) servers, and finally authoritative name servers responsible for individual domain zones. This hierarchical structure allows for efficient and scalable domain name resolution.

\begin{table}[H]
	\small
	\centering
	\caption{\textbf{Header flags format}}
	\begin{tabularx}{\linewidth}{|p{1cm}|p{0.597\linewidth}|p{1cm}|}
		\hline
		\textbf{Field} & \textbf{Description}                                                                                               & \textbf{Length (bits)} \\ \hline
		QR             & Indicates if the message is a query (0) or a reply (1)                                                             & 1                      \\ \hline
		OPCODE         & The type can be QUERY (standard query, 0), IQUERY (inverse query, 1), or STATUS (server status request, 2)         & 4                      \\ \hline
		AA             & Authoritative Answer, in a response, indicates if the DNS server is authoritative for the queried hostname         & 1                      \\ \hline
		TC             & TrunCation, indicates that this message was truncated due to excessive length                                      & 1                      \\ \hline
		RD             & Recursion Desired, indicates if the client means a recursive query                                                 & 1                      \\ \hline
		RA             & Recursion Available, in a response, indicates if the replying DNS server supports recursion                        & 1                      \\ \hline
		Z              & Zero, reserved for future use                                                                                      & 3                      \\ \hline
		RCODE          & Response code, can be NOERROR (0), FORMERR (1, Format error), SERVFAIL (2), NXDOMAIN (3, Nonexistent domain), etc. & 4                      \\ \hline
	\end{tabularx}
	\label{tab:query}
\end{table}

\subsection*{Query and Response}
The DNS protocol uses query and response messages to facilitate domain name resolution. A DNS query is initiated by a resolver and sent to a DNS server. The query message contains the requested domain name and the desired type of information (e.g., IP address, MX record). The DNS server processes the query and generates a response message containing the requested information, which is then returned to the resolver.
Each message consists of a header and four sections: question, answer, authority, and an additional space. A header field (flags) controls the content of these four sections.




\subsection*{Resource Records}
\begin{table}[H]
	\small
	\centering
	\caption{\textbf{Resource Record (RR) fields}}
	\begin{tabularx}{\linewidth}{|p{1.3cm}|p{0.558\linewidth}|p{1cm}|}
		\hline
		\textbf{Field} & \textbf{Description}                                                                                  & \textbf{Length (bits)} \\ \hline
		NAME           & Name of the node to which this record pertains                                                        & Variable               \\ \hline
		TYPE           & Type of RR in numeric form (e.g., 15 for MX RRs)                                                      & 16                     \\ \hline
		CLASS          & Class code                                                                                            & 16                     \\ \hline
		TTL            & Count of seconds that the RR stays valid (The maximum is $2^{31}-1$, which is approximately 68 years) & 32                     \\ \hline
		RDLENGTH       & Length of RDATA field (specified in octets)                                                           & 16                     \\ \hline
		RDATA          & Additional RR-specific data                                                                           & Variable               \\ \hline
	\end{tabularx}
	\label{tab:resource}
\end{table}

DNS resource records (RRs) are the building blocks of the DNS protocol. They contain specific information associated with a domain name. Common types of resource records include A records (IPv4 address), AAAA records (IPv6 address), MX records (mail exchange), CNAME records (canonical name), and NS records (name server). Each resource record has a specific format and contains relevant data for the DNS resolution process.




