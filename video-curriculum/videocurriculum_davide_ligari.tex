\documentclass[eng]{class}
% Publication Title
\title{Something about me}
% Short title for the header (copy the main title if it is not too long)
\shorttitle{Something about me}
       
% Authors
\author{D. Ligari 518592}
% Author Affiliations
\affil{Digital content retrieval, University of Pavia, Department of Computer Engineering (Data Science), Pavia, Italy}
% Surname of the first author of the manuscript
\videolink{https://youtu.be/NYCtO0WvBkE}
\firstauthor{D. Ligari}
%Contact Author Information
\contactauthor{D. Ligari} % Name and surname of the contact author
\email{davide.ligari01@universitadipavia.it} % Contact Author Email
% Publication data (will be defined in the edition)
\publicationdate{\today}
% Place your particular definitions here
\newcommand{\vect}[1]{\mathbf{#1}}  % vectors

\abstract{ This project focused on editing and organizing a videocurriculum as a real world project.
 A Work Breakdown Structure (WBS), Gantt chart, and risk analysis were utilized for effective management. 
 The WBS provided a hierarchical breakdown of tasks, facilitating coordination. The Gantt chart aided in scheduling, 
 resource allocation, and progress monitoring. 
 The risk analysis identified and mitigated potential threats. 
 By implementing these techniques, the project achieved efficiency and successful completion. 
 This report presents a systematic approach for managing similar complex projects, offering valuable insights for future endeavors.}
\keywords{ Video curriculum • Project management • WBS • GANTT • SWOT • Risk analysis}
\date{\today}
% Start document
\begin{document}
% Include title, authors, abstract, etc.
\maketitle
\tableofcontents
\thispagestyle{FirstPage}
%Figures and tables must be cited in the text and explained in detail. Do not forget to add a caption to each figure/table/
\section{Introduction}
\firstword{I}{n}
today's fast-paced and technologically driven world, the demand for innovative and engaging way to share the personal and working experiences has grown exponentially.
Traditional forms are increasingly being supplemented or even replaced by digital resources that leverage the power of multimedia.
Among these resources, video curriculum have emerged as a popular and effective alternative to the traditional curriculum.\\
A video curriculum is a short video that provides a concise overview of a person's professional and educational background.
The creation and management of a video curriculum, however, present unique challenges.
It requires meticulous planning, organization, and project management skills to ensure the seamless editing and effective organization of the video content.
Treating a video curriculum project as a complex endeavor can greatly enhance its success by providing a structured framework for its development.\\
This report focuses on the project management aspects of editing and organizing a video curriculum as if it were a real complex project.
It emphasizes the importance of employing project management techniques such as a Work Breakdown Structure (WBS), a Gantt chart,
and a risk analysis to ensure efficient execution and successful completion. By adopting these techniques,
the project team can effectively navigate the complexities inherent in video curriculum development,
resulting in a high-quality resource that meets the needs of recruiters.
\section{Work Breakdown Structure (WBS)}
\pagestyle{OtherPage}
\begin{figure}[b!]
  \centering
  \includegraphics[width=\textwidth]{images/wbs.png}
  \caption{WBS of the project}
  \label{fig-1}
\end{figure}
The Work Breakdown Structure (WBS) is a fundamental project management tool that provides a hierarchical breakdown of tasks and deliverables within a project.
In the context of video curriculum editing and organization, the WBS serves as a roadmap,
delineating the various activities and subtasks involved in the project's execution.
By breaking down the project into manageable components, the WBS enables effective coordination, resource allocation, and task management.\\
There are several types of WBS, including the deliverable-based, phase-based, and hybrid WBS.
In this project, a phase-based WBS was utilized, as it is the most commonly used and intuitive type, because it organizes tasks according to the  phases or stages of the project.
It provides a high-level overview of the project's progression and helps in managing tasks sequentially.\\
The Work Breakdown Structure (WBS) of the project, depicted in Figure 1, was organized into three levels.
Level 1 (green) represents the macro phases of the project, Level 2 (yellow) comprises the activities, and Level 3 (red) encompasses the sub-activities.
The project phases were arranged in a temporal sequence to facilitate effective management of the activities.
The project consisted of six macro-phases:\\
\\
\textbf{Requirement Analysis:}\\
This phase involved identifying the specifications imposed by the teacher for the video curriculum.
Additionally, deadlines for delivery (17 June) and timelines for each project phase were established.\\
\\
\textbf{Project Organization:}\\
During this phase, essential project management documents, including the WBS, GANTT chart, SWOT analysis, and risk analysis, were drafted.\\
\\
\textbf{Video Speech:}\\ This phase encompassed selecting topics to be covered in the video curriculum and preparing the speech to be delivered.
Corrections of pronunciation and grammatical errors were addressed.\\
\\
\textbf{Report:}\\ The report phase involved defining the general structure, topics, and relevant sections of the report.
This was followed by drafting the report and conducting a final revision.\\
\\
\textbf{Video Sharing:}\\
During this phase, video sharing platforms were analyzed, considering privacy concerns,
as the decision was made not to make the video curriculum public. The most suitable platform was chosen for sharing the video curriculum,
which underwent a final review before completion.
\clearpage
\section{GANTT Chart}
\begin{figure}[b!]
  \centering
  \includegraphics[width=0.9\textwidth]{images/gantt.png}
  \caption{Gantt chart}
  \label{fig-2}
\end{figure}
The Gantt chart is a widely adopted project management tool that provides a visual representation of project tasks, timelines, and dependencies.\\
In the context of the video curriculum editing and organization project, the Gantt chart played a critical role in planning, scheduling,
and monitoring project progress. By visually illustrating the project timeline and task interdependencies, the Gantt chart facilitated effective coordination,
resource allocation, and timely execution of project activities.
Figure \ref{fig-2} presents the  Gantt chart, which served as a visual representation of the project's timeline and activities.
Each phase of the project was further divided into sub-phases, allowing for a granular breakdown of tasks and deliverables.\\
A key aspect of developing the Gantt chart was to ensure the seamless coordination of activities by respecting
the dependencies between the various sub-phases.
For instance, the video recording phase was scheduled to commence only after the text had been written and corrected.
This approach aimed to streamline the workflow and promote efficient task progression, maximizing productivity and minimizing potential bottlenecks.\\
To provide a buffer for unforeseen challenges and potential delays, the deadline was set to 28 May.
This buffer period allowed for the mitigation of risks and provided an additional timeframe for addressing any unexpected obstacles that might arise.\\

\section{SWOT Analysis}
The SWOT analysis is a widely used strategic planning tool that helps organizations assess their internal strengths and weaknesses,
as well as external opportunities and threats. It provides a comprehensive framework for evaluating the current state of an organization or a project,
identifying areas of advantage and areas that require improvement, and uncovering potential opportunities and challenges in the external environment.\\
SWOT is an acronym that stands for Strengths, Weaknesses, Opportunities, and Threats. Strengths and weaknesses refer to internal factors within the organization,
such as resources, capabilities, processes, and performance. Opportunities and threats, on the other hand, are external factors that arise from the
business environment, market trends, competition, or regulatory changes.\\
\clearpage
\begin{figure}[b!]
  \centering
  \includegraphics[width=\textwidth]{images/SWOT.png}
  \caption{SWOT analysis}
  \label{fig-3}
\end{figure}
The purpose of conducting a SWOT analysis is to gain a deeper understanding of the organization's position,
make informed decisions, and develop strategies that leverage strengths, mitigate weaknesses, capitalize on opportunities,
and mitigate threats. By systematically evaluating these four dimensions, organizations can align their resources and efforts to achieve their
objectives and navigate through dynamic and competitive landscapes.\\
Figure \ref{fig-3} presents the SWOT analysis of the video curriculum project.

\section{Risk Analysis}
\section{Video creation and editing}

\section{Conclusions}

\pagestyle{OtherPage}

\end{document}