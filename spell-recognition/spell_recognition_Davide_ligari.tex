\documentclass{class}
\usepackage{blindtext}

\title{Speech Recognition}
\shortname{Speech Recognition}
\author[1]{Davide Ligari}
\affil[1]{University of Pavia, Department of Computer Engineering (Data Science), Pavia, Italy.\newline Course: Machine learning \newline Email: \url{davide.ligari01@universitadipavia.it}}

%% Abbreviated author list for the running footer
\runningauthor{D. Ligari}

\addbibresource{refs.bib}

\begin{document}

\maketitle

\begin{abstract}
  To be written

  \keywords{MLP Neural network • Features normalization • Confusion matrix • Speech recognition}
\end{abstract}
\section{MLP Neural network}
The Multilayer Perceptron (MLP) neural network is a popular type of artificial neural network used in machine learning.
It consists of interconnected layers of nodes or neurons that process data to produce predictions.
MLPs employ activation functions, such as sigmoid, tanh, or ReLU, to introduce non-linearity and capture complex patterns.
By adjusting the weights of these connections through a process called backpropagation, MLPs can learn from data and make accurate predictions.
They are widely used in tasks like image recognition, speech processing, and natural language understanding.





\section{Visualize the data}
\section{Batch size selection}
\section{Network architecture}
\subsection{Depth VS width}
\subsection{Choice of the optimal lambda}
\subsection{Optimal network}
\section{Model analysis}
\section{Feature normalization}
\section{Weights visualization}

\end{document}